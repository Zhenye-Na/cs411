%%%%%%%%%%%%%%%%%%%%%%%%%%%%%%%%%%%%%%%%%%%%%%%%%%%%%%%%%%%%%%%%%%%%%%
% LaTeX Example: Project Report
%
% Source: http://www.howtotex.com
%
% Feel free to distribute this example, but please keep the referral
% to howtotex.com
% Date: March 2011 
% 
%%%%%%%%%%%%%%%%%%%%%%%%%%%%%%%%%%%%%%%%%%%%%%%%%%%%%%%%%%%%%%%%%%%%%%
% How to use writeLaTeX: 
%
% You edit the source code here on the left, and the preview on the
% right shows you the result within a few seconds.
%
% Bookmark this page and share the URL with your co-authors. They can
% edit at the same time!
%
% You can upload figures, bibliographies, custom classes and
% styles using the files menu.
%
% If you're new to LaTeX, the wikibook is a great place to start:
% http://en.wikibooks.org/wiki/LaTeX
%
%%%%%%%%%%%%%%%%%%%%%%%%%%%%%%%%%%%%%%%%%%%%%%%%%%%%%%%%%%%%%%%%%%%%%%
% Edit the title below to update the display in My Documents
%\title{Project Report}
%
%%% Preamble
\documentclass[paper=a4, fontsize=11pt]{scrartcl}
\usepackage[T1]{fontenc}
\usepackage{fourier}
\usepackage{listings}
\usepackage[english]{babel}															% English language/hyphenation
\usepackage[protrusion=true,expansion=true]{microtype}	
\usepackage{amsmath,amsfonts,amsthm} % Math packages
\usepackage[pdftex]{graphicx}	
\usepackage{url}


%%% Custom sectioning
\usepackage{sectsty}
\allsectionsfont{\centering \normalfont\scshape}


%%% Custom headers/footers (fancyhdr package)
\usepackage{fancyhdr}
\usepackage{listings}
\pagestyle{fancyplain}
\fancyhead{}											% No page header
\fancyfoot[L]{}											% Empty 
\fancyfoot[C]{}											% Empty
\fancyfoot[R]{\thepage}									% Pagenumbering
\renewcommand{\headrulewidth}{0pt}			% Remove header underlines
\renewcommand{\footrulewidth}{0pt}				% Remove footer underlines
\setlength{\headheight}{13.6pt}


%%% Equation and float numbering
\numberwithin{equation}{section}		% Equationnumbering: section.eq#
\numberwithin{figure}{section}			% Figurenumbering: section.fig#
\numberwithin{table}{section}				% Tablenumbering: section.tab#


%%% Maketitle metadata
\newcommand{\horrule}[1]{\rule{\linewidth}{#1}} 	% Horizontal rule

\title{
	%\vspace{-1in} 	
	\usefont{OT1}{bch}{b}{n}
	\normalfont \normalsize \textsc{University of Illinois at Urbana-Champaign} \\ [25pt]
	\horrule{0.5pt} \\[0.4cm]
	\huge Assignment 3 - Report \\
	\horrule{2pt} \\[0.5cm]
}
\author{
	\normalfont 								\normalsize
	Department of Industrial and Enterprise Systems Engineering\\
	\normalsize Zhenye Na (zna2)\\[-3pt]		\normalsize
	\today
}
\date{}


%%% Begin document
\begin{document}
	\maketitle
	
	\section{Single-Relation Queries (30 pts)}
	
	\begin{enumerate}
		\item Consider the following relation:\\
		Graph(n1 , n2)\\
		A tuple (n1, n2) in Graph stores a directed edge from a node n1 to a node n2 in the corresponding graph. Your goal is to, for \textit{every} node in the graph, count the number of outgoing edges of that node. Note that for nodes without any outgoing edges, their edge count would be zero; you need to output this as well.
		You can assume that:
		\begin{enumerate}
			\item there are no duplicates or null values in the table; and
			\item every node in the graph is involved in at least one edge.
		\end{enumerate}
		
		\textbf{Solution: }
		\begin{lstlisting}
		SELECT n1, COUNT(n1)
		FROM Graph
		GROUP BY Graph.n1;
		UNION
		SELECT n2, 0
		FROM Graph as g1
		WHERE g1.n2 NOT IN (SELECT g2.n1
		FROM Graph as g2     
		);
		\end{lstlisting}
		
		
		\item Consider the following relation:\\
		Trained(\underline{student}, master, year)\\
		A tuple (S, M, Y) in Trained specifies that a SQL Master M trained student S who graduated in year Y. Your goal is to find \textit{the count of} SQL Masters who trained a student who graduated in the same year that 'Alice' or 'Bob' graduated.\\
		\textbf{Solution: }
		\begin{lstlisting}
		SELECT COUNT(DISTINCT t1.master) as CountofMaster
		FROM Trained as t1
		WHERE t1.year IN ( SELECT t2.year
		FROM Trained as t2
		WHERE t2.student = "Alice" 
		OR t2.student = "Bob"
		);
		\end{lstlisting}
		
		
		\item Consider the following relation:\\
		DBMS(\underline{operator}, system, performance)\\
		A tuple (O, S, P) in DBMS specifies an operator O in system S and has the performance value P. Your goal is to find those systems whose operators achieves a higher performance value on average than the average performance value in a system named 'PostgreSQL'.\\
		\textbf{Solution: }
		\begin{lstlisting}
		SELECT d3.system
		FROM ( SELECT d1.system as system, AVG(d1.performance) as performance
		FROM DBMS as d1
		GROUP BY d1.system
		) as d3
		WHERE d3.performance > ( SELECT AVG(d2.performance)
		FROM DBMS as d2
		WHERE d2.system = "PostgreSQL"
		);
		\end{lstlisting}
		
		
	\end{enumerate}
	
	
	\section{Multi-Relation Queries (20 pts)}
	Consider the following relations representing student information at UIUC:\\
	Mentorship(\underline{mentee\_sid}, mentor\_sid)\\
	Study(\underline{sid}, credits)\\
	Enrollment(\underline{did}, \underline{sid})\\
	Student(\underline{sid}, street , city)
	
	\begin{itemize}
		\item A tuple (M1, M2) in Mentorship specifies that M2 is a mentor of another student M1. 
		\item A tuple (S, C) in Study specifies that the student S has taken C credits.
		\item A tuple in Enrollment (D, S) specifies that student S is enrolled in department D. 
		\item A (ST, S, C) in Student specifies that student ST lives on street S in city C.
	\end{itemize}
	
	\begin{enumerate}
		\item Find all students who live in the same city and on the same street as their mentor.\\
		\textbf{Solution: }
		\begin{lstlisting}
		SELECT m1.mentee_sid
		FROM Mentorship as m1, Student as s1, Student as s2
		WHERE m1.mentee_sid = s1.sid
		AND m1.mentor_sid = s2.sid
		AND s1.street = s2.street
		AND s1.city = s2.city;
		\end{lstlisting}
		
		
		\item Find all students(i.e., distinct sid) who have taken more credits than the average credits of all of the students of their department.\\
		\textbf{Solution: }
		\begin{lstlisting}
		SELECT s1.sid
		FROM Study as s1, Enrollment as e1
		WHERE e1.sid = s1.sid
		AND s1.credits > ( SELECT AVG(s2.credits)
		FROM Study as s2, Enrollment as e2
		WHERE s2.sid = e2.sid
		AND e1.did = e2.did
		);
		\end{lstlisting}
		
		
	\end{enumerate}
	
	
	\section{Database Manipulation and Views (25 pts)}
	
	\begin{enumerate}
		\item In the Study relation, insert a new student, whose id is 66666 and has 0 credits.\\
		\textbf{Solution: }
		\begin{lstlisting}
		INSERT INTO Study
		VALUES('66666', '0');
		\end{lstlisting}
		
		\item In the Study relation, delete students who have graduated (i.e., the ones who have more than 200 credits).\\
		\textbf{Solution: }
		\begin{lstlisting}
		DELETE FROM Study
		WHERE Study.credits > 200;
		\end{lstlisting}
		
		\item In the Study relation, add 2 credits for students who are mentors.\\
		\textbf{Solution: }
		\begin{lstlisting}
		UPDATE Study
		SET Study.credits = Study.credits + 2
		WHERE Study.sid IN ( SELECT m1.mentor_sid
		FROM Mentorship as m1
		);
		\end{lstlisting}
		
		\item Incoming students are those who have been accepted (i.e., exist in the \textbf{Student} relation) but have not registered in any department (i.e., do not exist in the \textbf{Enrollment} relation). Create a View that contains \textbf{sid} of all incoming students.\\
		\textbf{Solution: }
		\begin{lstlisting}
		CREATE VIEW Incoming AS
		SELECT s1.sid
		FROM Student as s1
		WHERE s1.sid NOT IN ( SELECT e1.sid
		FROM Enrollment as e1
		);
		\end{lstlisting}
		
	\end{enumerate}
	
	
	
	\section{Constraints and Triggers (25 pts)}
	\begin{enumerate}
		\item Consider the following relation: \\
		Payment(salary , bonus) \\
		Write a schema-level assertion using the "CREATE ASSERTION" statement to ensure that no bonus is larger than the maximum salary in the \textbf{Payment} relation.\\
		\textbf{Solution: }
		\begin{lstlisting}
		CREATE ASSERTION Bonus
		CHECK(
		NOT EXISTS( SELECT Payment.bonus
		FROM Payment 
		WHERE Payment.bonus > MAX(Payment.salary)
		));
		\end{lstlisting}
		
		\item Consider the following relation: \\
		Study(\underline{sid}, major , GPA) \\
		Write a trigger T1 that increases the GPA by 10\% for those students who transform their major from any Non-CS major to 'CS'.\\
		\textbf{Solution: }
		\begin{lstlisting}
		CREATE TRIGGER T1
		AFTER UPDATE OF major ON Study
		REFERENCING
		OLD ROW AS old
		NEW ROW AS new
		FOR EACH ROW
		WHEN(
		old.major <> 'CS'
		AND new.major = 'CS'
		)
		UPDATE Study
		SET new.GPA = old.GPA * 1.1;
		
		
		\end{lstlisting}
		
		
		
	\end{enumerate}
	
	
	
	
	
	%%% End document
\end{document}